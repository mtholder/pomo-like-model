% This is LLNCS.DEM the demonstration file of
% the LaTeX macro package from Springer-Verlag
% for Lecture Notes in Computer Science,
% version 2.4 for LaTeX2e as of 16. April 2010
%
\documentclass{llncs}
%
\usepackage{amssymb}
\usepackage{amsmath}
\usepackage{pdfpages}
\usepackage{graphicx}
\usepackage{paralist}
\usepackage{xspace}
\usepackage{paralist}
\usepackage{bm}
\usepackage{algorithm,algorithmic}
\newcommand{\numOTUs}{\ensuremath{L}}
\newcommand{\numSites}{\ensuremath{K}}
\newcommand{\numSp}{\ensuremath{M}}
\newcommand{\numObsStates}{\ensuremath{k}}
\newcommand{\virtPopSize}{\ensuremath{N}}
\newcommand{\pomo}{PoMo\xspace}
% from http://tex.stackexchange.com/a/33547
\newcommand{\appropto}{\mathrel{\vcenter{
              \offinterlineskip\halign{\hfil$##$\cr
                      \propto\cr\noalign{\kern2pt}\sim\cr\noalign{\kern-2pt}}}}}
\DeclareMathOperator*{\argmax}{\arg\!\max}
\usepackage{hyperref}
\hypersetup{backref,  linkcolor=blue, citecolor=red, colorlinks=true, hyperindex=true}
%
\begin{document}
\title{POMO implementation}
\titlerunning{POMO implementation}
\author{Author 1\inst{1} \and Author 2\inst{1} \and Author 3\inst{1,2}}
\authorrunning{Author 1 et al.} % abbreviated author list
\tocauthor{Author 1, Author 2}
\institute{Institute 1\\
\email{\{Authors\}@h-its.org} \and Institute 2,\\ Address, Country\\
\email{Author3@h-its.org}}
%% Eumerating parts in aligned environments %%
\newcommand\enum{\addtocounter{equation}{1}\tag{\theequation}}
\maketitle              % typeset the title of the contribution
\section {Introduction}
DeMaio, Schl\"otterer, and Kosiol \cite{DeMaioSK2013} introduced a clever approach to dealing with polymorphism in the context of
    phylogenetic estimation.
Rather than explicitly treating the gene genealogy of a locus as a nuisance parameter, their polymorphism-aware
    phylogenetic model (\pomo) uses hidden state approach to model the frequency of residues for a site.

Input: Each of $\numOTUs$ OTUs is assumed to have sequence data from $\numSites$ sites and be mapped to one of $\numSp$ species.

The number of states of the data is $\numObsStates$. 
The model is described for DNA data ($\numObsStates=4$), but could be used for other small values of $\numObsStates$ if
    the process of fixation of a new state is assumed to be similiar to the genetic drift/selection as modelled by \pomo.

The model treats the evolution of each site as a continuous-time Markov process with a larger state space which
    includes states that correspond to the species being monomorphic for each of the $\numObsStates$ but als
    considers states representing polymorphic states in which 2 alleles occur at differing frequencies.
The continuous frequency of one allele in a population is binned for computational convenience.
This is done by imagining a virtual population of size $\virtPopSize$.

The process one residue replacing another would consists of:
\begin{compactenum}
\item A mutation such that the new allele has frequency $1/\virtPopSize$,
\item pass through each of the other polymorphic bins for this pair, $\{2/\virtPopSize, 3/\virtPopSize,\ldots,(\virtPopSize-1)/\virtPopSize, 1\}$
    where a frequency of 1 corresponds to the state that maps to the new allele's ``observable'' state being fixed.
\end{compactenum}
The drift process is reversible.

Evolutionary trajectories that involve a species being polymorphic for than one allele are clearly 
    biologically possible, but are prohibited to reduce the size of the state space.
    The state space of evolution in \pomo is: $\numObsStates + {\numObsStates \choose 2} \left(\virtPopSize - 1\right)$
\bibliographystyle{splncs03}
\bibliography{pomo}



\end{document}

\begin{algorithm} \caption{}\label{}
\begin{algorithmic}
\end{algorithmic}
\end{algorithm}
