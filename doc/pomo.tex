% This is LLNCS.DEM the demonstration file of
% the LaTeX macro package from Springer-Verlag
% for Lecture Notes in Computer Science,
% version 2.4 for LaTeX2e as of 16. April 2010
%
\documentclass{llncs}
%
\usepackage{amssymb}
\usepackage{amsmath}
\usepackage{pdfpages}
\usepackage{graphicx}
\usepackage{paralist}
\usepackage{xspace}
\usepackage{paralist}
\usepackage{bm}
\usepackage{algorithm,algorithmic}
\newcommand{\numOTUs}{\ensuremath{L}}
\newcommand{\numSites}{\ensuremath{K}}
\newcommand{\numSp}{\ensuremath{M}}
\newcommand{\numObsStates}{\ensuremath{k}}
\newcommand{\virtPopSize}{\ensuremath{N}}
\newcommand{\polyProb}{\ensuremath{\pi_{\mbox{pol}}}}
\newcommand{\pomo}{PoMo\xspace}
% from http://tex.stackexchange.com/a/33547
\newcommand{\appropto}{\mathrel{\vcenter{
              \offinterlineskip\halign{\hfil$##$\cr
                      \propto\cr\noalign{\kern2pt}\sim\cr\noalign{\kern-2pt}}}}}
\DeclareMathOperator*{\argmax}{\arg\!\max}
\usepackage{hyperref}
\hypersetup{backref,  linkcolor=blue, citecolor=red, colorlinks=true, hyperindex=true}
%
\begin{document}
\title{A \pomo-like model for large-scale phylogenetics}
\titlerunning{POMO implementation}
\author{Mark T.~Holder\inst{1,2}\and Alexandros Stamatakis\inst{1,3}}
\authorrunning{Author 1 et al.} % abbreviated author list
%\tocauthor{Author 1, Author 2}
\institute{1. HITS gGmbH,\\2. Univ.~Kansas\\3. KIT}\\
%\email{\{Authors\}@h-its.org} \and Institute 2,\\ Address, Country\\
%\email{Author3@h-its.org}}
%% Eumerating parts in aligned environments %%
\newcommand\enum{\addtocounter{equation}{1}\tag{\theequation}}
\maketitle              % typeset the title of the contribution
\section {Motivation}
Large scale phylogenetic studies feature large numbers of species sampled with relatively few
    individuals sampled per species.
This sampling strategy make it difficult to examine the history of gene geneaologies with great precision, but 
    discordance between gene trees and species trees motivates the need for developing methods
    that recognize that polymorphism at the time of speciation can result in patterns of fixation
    that look like homoplasy when mapped onto the species tree.

The \pomo model of \cite{DeMaioSK2013} provides an elegant approximate method for dealing with polymorphism 
    on species trees.
However, the model is more general that needed for many phylogenetic purposes\footnote{or at least we {\em hope} that
    one need not implement such a rich model for the purpose of estimating trees in most cases.} and
    they formulate the model without the constraint of time reversibility.
The lack of time reversibility adds realism, but necessitates the use of rooted tree representations (furthermore
    the infer new parameters to describe the frequency of states at the root).

Here we develop a time-reversible that borrows ideas from the \pomo model, but is more amenable to 
    implementation in ExaML \cite{ExaMLInitial,ExaMLLatest}.
\section {Introduction}
DeMaio, Schl\"otterer, and Kosiol \cite{DeMaioSK2013} introduced a clever approach to dealing with polymorphism in the context of
    phylogenetic estimation.
Rather than explicitly treating the gene genealogy of a locus as a nuisance parameter, their polymorphism-aware
    phylogenetic model (\pomo) uses hidden state approach to model the frequency of residues for a site.

Input: Each of $\numOTUs$ OTUs is assumed to have sequence data from $\numSites$ sites and be mapped to one of $\numSp$ species.

The number of states of the data is $\numObsStates$. 
The model is described for DNA data ($\numObsStates=4$), but could be used for other small values of $\numObsStates$ if
    the process of fixation of a new state is assumed to be similiar to the genetic drift/selection as modelled by \pomo.

The model treats the evolution of each site as a continuous-time Markov process with a larger state space which
    includes states that correspond to the species being monomorphic for each of the $\numObsStates$ but als
    considers states representing polymorphic states in which 2 alleles occur at differing frequencies.
The continuous frequency of one allele in a population is binned for computational convenience.
This is done by imagining a virtual population of size $\virtPopSize$.

The process one residue replacing another would consists of:
\begin{compactenum}
\item A mutation such that the new allele has frequency $1/\virtPopSize$,
\item pass through each of the other polymorphic bins for this pair, $\{2/\virtPopSize, 3/\virtPopSize,\ldots,(\virtPopSize-1)/\virtPopSize, 1\}$
    where a frequency of 1 corresponds to the state that maps to the new allele's ``observable'' state being fixed.
\end{compactenum}
The drift process is reversible.

Evolutionary trajectories that involve a species being polymorphic for than one allele are clearly 
    biologically possible, but are prohibited to reduce the size of the state space.
    The state space of evolution in \pomo is: $\numObsStates + {\numObsStates \choose 2} \left(\virtPopSize - 1\right)$

\section{notation}
Just reiterating this from  \cite{DeMaioSK2013} for convenience:
\begin{compactitem}
\item[$\bullet$] $\virtPopSize$ the virtual population size.
\item[$\bullet$] $M_{IJ}^{i,k}$ the instantaneous rate of change from having the state corresponding to having $i$ copies of allele $I$ and $N-i$ copeies of $J$ to having $k$ copies of $I$ and $N-k$ copies of $J$
\item[$\bullet$] $\polyProb$ is the {\em a priori} probability of being in a polymorphic state at the root.
\item[$\bullet$] $s_I$ the selection coefficient of the allele with state $I$.
\end{compactitem}

\section{A time-reversible form of \pomo}
The constraint of time-reversibility is that, $\forall X \neq Y$:
\begin{eqnarray}
    \pi_X q_{XY} = \pi_Y q_{YX}
\end{eqnarray}
were $\pi_X$ is the equilibrium state frequency of $X$, and $q_{XY}$ is the instantaneous
    rate of that an element of state $X$ tranitions to one of state $Y$.
This enables the refactoring into:
\begin{eqnarray}
    \pi_X q_{XY} & = & \pi_Y q_{YX} \\
    q_{XY} & = & r_{XY}\pi_Y \\
    q_{YX} & = & r_{XY}\pi_X
\end{eqnarray}
to emphasize the existence of a symmetric factor in the rate matrix, $r_{XY}$, multipled by the destination state frequency.

We can most clearly see that \pomo does not fit easily into a time-reversible framework by considering 
    the transition $M_{IJ}^{0,1}$ vs $M_{IJ}^{1,0}$. 
The former should be a very low rate constant reflecting the mutation rate.
The latter involves the fixation of an allele at very high frequency.
As Table S9 of  \cite{DeMaioSK2013} shows, these rates have different functional forms with respect to $N$ (the former is $N^2$; the latter is a function of $N^{-1}$).

Nevertheless, we can create \pomo-inspired reversible model by using
    $s_I= s_J$ for all $I$ and $J$, and then using the Eqn (4) of \cite{DeMaioSK2013}
    as the basis for the equilibrium state frequencies:
\begin{eqnarray}
\pi_{IJ}^i & = & \frac{\polyProb}{K_{\mbox{norm}}}\left(
    \frac{\pi_J\mu_{JI}}{i} + \frac{\pi_I\mu_{IJ}}{N-i}\right) \hskip 4em \forall i: 0<i<N \label{stateFreqPoly}\\
    K_{\mbox{norm}} & = & \sum_I \sum_{J\neq I}\sum_{i=1}^{N-1} \left(
    \frac{\pi_J\mu_{JI}}{i} + \frac{\pi_I\mu_{IJ}}{N-i}\right)
\end{eqnarray}
with the equilibrium frequency of being monomorphic for state $I$ is
$\pi_I(1-\polyProb)$ as assumed for the root state frequencies (on page 2259 of \cite{DeMaioSK2013}).

Setting $s_I=s_J$ simplifies eqn (1) and (2) of \cite{DeMaioSK2013} to:
\begin{eqnarray}
 $M_{IJ}^{i,i+1}$  = $M_{IJ}^{i,i-1}$ & = & \frac{i(N-i)}{N^2}
\end{eqnarray}

If we further think of the state frequencies being driven by a mutation rate composed of a product
    the nucleotide relative frequency and symmetric relative rate of interchange:
\begin{eqnarray}
 \mu_{IJ} & = &\pi_J r_{IJ} \\
 \mu_{JI} & = &\pi_I r_{IJ} 
\end{eqnarray}
then equation \ref{stateFreqPoly} becomes:
\begin{eqnarray}
\pi_{IJ}^i & = & \frac{\polyProb}{K_{\mbox{norm}}}\left( \frac{\pi_J\pi_I r_{IJ}}{i} + \frac{\pi_I\pi_J r_{IJ}}{N-i}\right) \nonumber\\
      & = & \frac{\polyProb}{K_{\mbox{norm}}}\left( \frac{N\pi_I\pi_J\mu_{IJ}}{i(N-i)}\right) \nonumber
\end{eqnarray}

An MLE of $\polyProb$ could not be reliably estimated by \cite{DeMaioSK2013}, presumably because it only
    affects the likelihood via its affect on the weighting of states at the root.



 


\bibliographystyle{splncs03}
\bibliography{pomo}




\end{document}

\begin{algorithm} \caption{}\label{}
\begin{algorithmic}
\end{algorithmic}
\end{algorithm}
