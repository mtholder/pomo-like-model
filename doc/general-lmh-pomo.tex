% This is LLNCS.DEM the demonstration file of
% the LaTeX macro package from Springer-Verlag
% for Lecture Notes in Computer Science,
% version 2.4 for LaTeX2e as of 16. April 2010
%
\documentclass{llncs}
%
\usepackage{amssymb}
\usepackage{amsmath}
\usepackage{pdfpages}
\usepackage{graphicx}
\usepackage{paralist}
\usepackage{xspace}
\usepackage{paralist}
\usepackage{bm}
\usepackage{algorithm,algorithmic}
\setlength{\parindent}{0}
\setlength{\parskip}{0.5em}
\newcommand{\numOTUs}{\ensuremath{L}}
\newcommand{\numSites}{\ensuremath{K}}
\newcommand{\numSp}{\ensuremath{M}}
\newcommand{\numObsStates}{\ensuremath{k}}
\newcommand{\virtPopSize}{\ensuremath{N}}
\newcommand{\polyProb}{\ensuremath{\phi}}}
\newcommand{\pomoState}[3]{\ensuremath{\mathcal{S}_{#1#2}^{#3}}}}
\newcommand{\triPomoState}[5]{\ensuremath{\mathcal{S}_{#1#2#3}^{#4,#5}}}}
\newcommand{\quadPomoState}[3]{\ensuremath{\mathcal{S}_{\texttt{ACGT}}^{#1,#2,#3}}}}
\newcommand{\Knorm}{\ensuremath{K}}}
\newcommand{\Klmh}{\ensuremath{K_{LMH}}}}
\newcommand{\pomo}{PoMo\xspace}
% from http://tex.stackexchange.com/a/33547
\newcommand{\appropto}{\mathrel{\vcenter{
              \offinterlineskip\halign{\hfil$##$\cr
                      \propto\cr\noalign{\kern2pt}\sim\cr\noalign{\kern-2pt}}}}}
\DeclareMathOperator*{\argmax}{\arg\!\max}
\usepackage{hyperref}
\hypersetup{backref,  linkcolor=blue, citecolor=red, colorlinks=true, hyperindex=true}
%
\begin{document}
\title{A general implementation of the lmh-PoMo-like model}
\titlerunning{POMO implementation}
\author{Mark T.~Holder\inst{1,2}\and Alexandros Stamatakis\inst{1,3}}
\authorrunning{Author 1 et al.} % abbreviated author list
%\tocauthor{Author 1, Author 2}
\institute{1. HITS gGmbH,\\2. Univ.~Kansas\\3. KIT}\\
%\email{\{Authors\}@h-its.org} \and Institute 2,\\ Address, Country\\
%\email{Author3@h-its.org}}
%% Eumerating parts in aligned environments %%
\newcommand\enum{\addtocounter{equation}{1}\tag{\theequation}}
\maketitle              % typeset the title of the contribution
\section {Motivation}
This document is an extension of:
 \url{https://github.com/mtholder/pomo-like-model/blob/master/doc/lmh-pomo.tex}
that was pulled out for the sake of briefer presentation.

In most phylogenetic datasets we do not anticipate having a lot of info
    on polymorophism.
Nor do we expect to see many cases in which a species shows more than 2 alleles
    at a site.

Nevertheless, the fact that PoMo would assign a likelihood of 0 to such cases, is
    certainly a hurdle to use the general use of the model.

Following the spirit of the data-driven approach described in the lmh-pomo.tex document,
    this document will describe an model with only tenuous connections to the relevant
    population genetics theory.


\subsection*{Notation}
\begin{compactenum}
    \item $N$ is the size of the virtual population.
    \item $\pi_I$ is the frequency of nucleotide $I$ in the mutational model.
    \item $r_{IJ}$ is the symmetric factor in the mutational model. So $\mu_{IJ} = \pi_J r_{IJ}$ and $\mu_{JI} = \pi_I r_{IJ}$
    \item $\pi_{w,x,y,z}$ is the equilibrium frequency of a PoMo state that corresponds to
        $w$ members of the virtual population having of \texttt{A}, 
        $x$ members of the virtual population having of \texttt{C},
        $y$ members of the virtual population having of \texttt{G}, 
        $z$ members of the virtual population having of \texttt{T}. Note that $0\leq w, x, y, z$ and
        $z = N- w- x-y$.
    \item $r_{w,x,y,z}^{w^\prime,x^\prime,y^\prime,z^\prime}$ denotes the symmetric factor of the instantaneous rate associated with moving with a virtual population count vector $(w,x,y,z)\leftrightarrow\left({w^\prime,x^\prime,y^\prime,z^\prime}\right)$
    \item $q_{w,x,y,z}^{w^\prime, x^\prime, y^\prime, z^\prime}$ denotes the
    instantaneous rate of with moving with a virtual population count vector $(w,x,y,z)\rightarrow\left({w^\prime,x^\prime,y^\prime,z^\prime}\right)$.
    By time reversibility:
    $q_{w,x,y,z}^{w^\prime, x^\prime, y^\prime, z^\prime} = \pi_{w^\prime, x^\prime, y^\prime, z^\prime} r_{w,x,y,z}^{w^\prime,x^\prime,y^\prime,z^\prime}$
\end{compactenum}

\section*{General-LMH-PoMo}
\begin{compactenum}
    \item $N$ is a fixed-by-the-user virtual population size.
    \item In the diallelic condition, the first allele in the subscript is binned into $l$ (low = $1/N$), $m_2$ (mid=0.5), and $h_2$ (high= $(N-1)/N$) frequencies
    \item In the 3-allelic condition, the first allele in the subscript is binned into $l$ (low = $1/N$), $m_3$ (mid=1/3), and $h_3$ (high= $(N-2)/N$) frequencies
    \item In the 4-allelic condition, the first allele in the subscript is binned into $l$ (low = $1/N$), $m_4$ (mid=0.25), and $h_4$ (high= $(N-3)/N$) frequencies
\end{compactenum}


This would yield 43 states:
\begin{compactenum}
    \item The 4 monomorphic states: $(N,0,0,0)$, $(0,N,0,0)$, $(0,0,N,0)$, and $(0,0,0,N)$.
    \item The 18 dimorphic states. For each of the 6 pairs of nucleotides, there would be 3 polymorphic bins such as:
    $(N-1, 1, 0, 0)$, $(N/2,N/2,0,0)$, and $(1, N-1, 0, 0)$.
    \item There are 16 states with 3 alleles. For each of the 4 triples of nucleotides, there would be 4 states:
    $(N-2, 1, 1, 0)$, $(1, N-2, 1, 0)$, $(1, 1, N-2, 0)$, and $(N/3, N/3, N/3, 0)$.
    \item There are 4 states with all 4 alleles:
    $(N-3, 1, 1, 1)$, $(1, N-3, 1, 1)$, $(1, 1, N-3, 1)$, $(1, 1, 1, N-3)$ and $(N/4, N/4, N/4, N/4)$.
\end{compactenum}



\subsection*{Parameterization}

This coarse binning breaks the easy connections with the population genetics underpinning.
However, we expect to have quite a bit of data, so we
could estimate an extra parameter to generate the state frequencies, and $Q$ matrix:
\begin{compactitem}
    \item The 3 free parameters needed to yield the 4 equilibrium frequency of the mutation process: $1 = \pi_A + \pi_C + \pi_G + \pi_T$
    \item The 5 free parameters needed to yield the 6 symmetric factors of the GTR rate matrix:  $\{r_{AC}$, $r_{AG}$, $r_{AT}$, $r_{CG}$, $r_{CT}$, $r_{GT}\}$
    \item $[\phi_1, \phi_2, \phi_3, \phi_4]$ sum to 1 and with $\phi_i$ representing the probability of being
    a state that corresponds to the species showing $i$ states. 
    \item The relative frequency parameter, $\psi = \pi_{IJ}^{(N/2)}/\pi_{IJ}^{(N-1)} = \pi_{IJ}^{(N/2)}/\pi_{IJ}^{(1)}$ is used as in lmh-PoMo.
    This govens the relative abundance of the mid-frequency bin states, given that 
    the state corresponds to a polymorphic condition.
    \item a new symmetric rate factor: $\rho$ that governs the relative rate of drift between states that correspond to polymorphic for the same pair of nucleotides
\end{compactitem}

If $\phi_3 = \phi_4 = 0$ then $\phi_1, = 1- \phi_2$, this model reduces to the lmh-PoMo

\subsection*{Summarizing $Q$ matrix and state frequency calculations for LMH-PoMo}

Monomorphic state freq: 
\begin{equation}
    \pi_{IJ}^{(N)} = \pi_{JI}^{(0)} = \pi_I(1-\polyProb)
\end{equation}

The normalizing constant for comparing the frequencies of different polymorphic states becomes:
\begin{equation}
    \Klmh = \sum_I\sum_{J > I}\pi_I\pi_J r_{IJ}
\end{equation}
such that:
    $$\Pr(\mbox{polymorphic for }I, J \mid \mbox{polymorphic}) = \frac{\pi_I\pi_J r_{IJ}}{\Klmh}$$

By the definition of the parameter $\phi$, this leads to:
    $$\Pr(\mbox{polymorphic for }I, J) = \frac{\phi\pi_I\pi_J r_{IJ}}{\Klmh}$$

Given that the system is in a polymorphic condition, the probablity of being in a low frequency bin is 
    $$\Pr(\mbox{freq =} 1 \mid \mbox{polymorphic}) = \Pr(\mbox{freq =} N -1 \mid \mbox{polymorphic}) = \frac{1}{2 + \psi}.$$
This is also the probability of being in a high frequency bin.
The corresponding probability for the being assigned to the mid-frequency bin is 
$$\Pr(\mbox{freq =} N/2 \mid \mbox{polymorphic}) =  \frac{\psi}{2 + \psi}.$$

Thus, the equilibrium frequency of a low- frequency diallelic state is: 
\begin{equation}
    \pi_{IJ}^{(1)} = \pi_{IJ}^{(N-1)}  =  \frac{\polyProb\pi_I\pi_J r_{IJ}}{\Klmh (2 + \psi)}
\end{equation}

The mid-frequency diallelic state: 
\begin{equation}
    \pi_{IJ}^{(N/2)} = \frac{\polyProb\pi_I\pi_J r_{IJ}\psi}{\Klmh (2 + \psi)} 
\end{equation}


All elements of $Q$ are 0 between ``non-adjacent'' states are 0.

The diagonal element of the $Q$ matrix is the negative sum of the other elements.

The rate of mutation to introduce state $J$:
\begin{equation}
   Q_{IJ}^{N,N-1} = \mu_{IJ} = \pi_J r_{IJ}.
\end{equation}
In PoMo this includes a factor of $N$, but in our parameterization with $\phi$ as a free parameter, the effect of $N$ simply changes the MLE of $\phi$.

Exploiting time-reversibility:
\begin{eqnarray}
    Q_{IJ}^{N,N-1} & = & \pi_J r_{IJ} \nonumber\\
    & = & \pi_{IJ}^{(N-1)} r_{IJ}^{N, N-1} \nonumber\\
\end{eqnarray}
which enables us to solve for the reversible factor in the introduction/loss of polymorphism
\begin{eqnarray}
  r_{IJ}^{N, N-1} & = & \frac{\pi_J r_{IJ}}{ \pi_{IJ}^{(N-1)}} \nonumber\\
  & = & \frac{\pi_J r_{IJ}\Klmh (2 + \psi)}{ \polyProb\pi_I\pi_J r_{IJ}} \nonumber \\
  & = & \frac{\Klmh (2 + \psi)}{ \polyProb\pi_I}
\end{eqnarray}

So the loss of  $J$ to become monomorphic for state $I$ is
\begin{eqnarray}
   Q_{IJ}^{N-1,N} & = & r_{IJ}^{N, N-1} \pi_{IJ}^{(N)} \nonumber \\
    & = & \frac{\Klmh (2 + \psi)}{ \polyProb\pi_I} \pi_I (1 -\phi) \nonumber \\
    & = & \frac{(1 -\phi) \Klmh (2 + \psi)}{ \polyProb}
\end{eqnarray}

The transitions between the polymorphic states is determined by the new drift parameter:
\begin{eqnarray}
   Q_{IJ}^{N-1,N/2} = Q_{IJ}^{1,N/2} & = & \frac{\polyProb\pi_I\pi_J r_{IJ}\psi \rho}{\Klmh (2 + \psi)} \\
   Q_{IJ}^{N/2,N-1} = Q_{IJ}^{N/2,1} & = & \frac{\polyProb\pi_I\pi_J r_{IJ} \rho}{\Klmh (2 + \psi)} 
\end{eqnarray}




\bibliographystyle{splncs03}
\bibliography{pomo}




\end{document}

\begin{algorithm} \caption{}\label{}
\begin{algorithmic}
\end{algorithmic}
\end{algorithm}
