% This is LLNCS.DEM the demonstration file of
% the LaTeX macro package from Springer-Verlag
% for Lecture Notes in Computer Science,
% version 2.4 for LaTeX2e as of 16. April 2010
%
\documentclass{llncs}
%
\usepackage{amssymb}
\usepackage{amsmath}
\usepackage{pdfpages}
\usepackage{graphicx}
\usepackage{paralist}
\usepackage{xspace}
\usepackage{paralist}
\usepackage{bm}
\usepackage{algorithm,algorithmic}
\setlength{\parindent}{0}
\setlength{\parskip}{0.5em}
\newcommand{\numOTUs}{\ensuremath{L}}
\newcommand{\numSites}{\ensuremath{K}}
\newcommand{\numSp}{\ensuremath{M}}
\newcommand{\numObsStates}{\ensuremath{k}}
\newcommand{\virtPopSize}{\ensuremath{N}}
\newcommand{\polyProb}{\ensuremath{\phi}}}
\newcommand{\pomoState}[3]{\ensuremath{\mathcal{S}_{#1#2}^{#3}}}}
\newcommand{\triPomoState}[5]{\ensuremath{\mathcal{S}_{#1#2#3}^{#4,#5}}}}
\newcommand{\quadPomoState}[3]{\ensuremath{\mathcal{S}_{\texttt{ACGT}}^{#1,#2,#3}}}}
\newcommand{\Knorm}{\ensuremath{K}}}
\newcommand{\Klmh}{\ensuremath{K_{LMH}}}}
\newcommand{\pomo}{PoMo\xspace}
% from http://tex.stackexchange.com/a/33547
\newcommand{\appropto}{\mathrel{\vcenter{
              \offinterlineskip\halign{\hfil$##$\cr
                      \propto\cr\noalign{\kern2pt}\sim\cr\noalign{\kern-2pt}}}}}
\DeclareMathOperator*{\argmax}{\arg\!\max}
\usepackage{hyperref}
\hypersetup{backref,  linkcolor=blue, citecolor=red, colorlinks=true, hyperindex=true}
%
\begin{document}
\title{A general implementation of the lmh-PoMo-like model}
\titlerunning{POMO implementation}
\author{Mark T.~Holder\inst{1,2}\and Alexandros Stamatakis\inst{1,3}}
\authorrunning{Author 1 et al.} % abbreviated author list
%\tocauthor{Author 1, Author 2}
\institute{1. HITS gGmbH,\\2. Univ.~Kansas\\3. KIT}\\
%\email{\{Authors\}@h-its.org} \and Institute 2,\\ Address, Country\\
%\email{Author3@h-its.org}}
%% Eumerating parts in aligned environments %%
\newcommand\enum{\addtocounter{equation}{1}\tag{\theequation}}
\maketitle              % typeset the title of the contribution
\section {Motivation}
This document is an extension of:\\
 \url{https://github.com/mtholder/pomo-like-model/blob/master/doc/lmh-pomo.tex}\\
that was pulled out for the sake of briefer presentation.

In most phylogenetic datasets we do not anticipate having a lot of info
    on polymorophism.
Nor do we expect to see many cases in which a species shows more than 2 alleles
    at a site.
Nevertheless, the fact that PoMo would assign a likelihood of 0 to such cases, is
    certainly a hurdle to use the general use of the model.

Following the spirit of the data-driven approach described in the lmh-pomo.tex document,
    this document will describe an model with only tenuous connections to the relevant
    population genetics theory.

\subsection*{Notation}
\begin{compactenum}
    \item $N$ is the size of the virtual population.
    \item $\pi_I$ is the frequency of nucleotide $I$ in the mutational model.
    \item $r_{IJ}$ is the symmetric factor in the mutational model.
    So $\mu_{IJ} = \pi_J r_{IJ}$ and $\mu_{JI} = \pi_I r_{IJ}$
    \item $\pi_{w,x,y,z}$ is the equilibrium frequency of a PoMo state that corresponds to
        $w$ members of the virtual population having of \texttt{A}, 
        $x$ members of the virtual population having of \texttt{C},
        $y$ members of the virtual population having of \texttt{G}, 
        $z$ members of the virtual population having of \texttt{T}. Note that $0\leq w, x, y, z$ and
        $z = N- w- x-y$.
    \item $r_{w,x,y,z}^{w^\prime,x^\prime,y^\prime,z^\prime}$ denotes
    the symmetric factor of the instantaneous rate associated with moving with a virtual population count 
    vector $(w,x,y,z)\leftrightarrow\left({w^\prime,x^\prime,y^\prime,z^\prime}\right)$
    \item $q_{w,x,y,z}^{w^\prime, x^\prime, y^\prime, z^\prime}$ denotes the
    instantaneous rate of with moving with a virtual population count 
    vector $(w,x,y,z)\rightarrow\left({w^\prime,x^\prime,y^\prime,z^\prime}\right)$.
    By time reversibility:
    $q_{w,x,y,z}^{w^\prime, x^\prime, y^\prime, z^\prime} = \pi_{w^\prime, x^\prime, y^\prime, z^\prime} r_{w,x,y,z}^{w^\prime,x^\prime,y^\prime,z^\prime}$
\end{compactenum}

\section*{General-LMH-PoMo}
\begin{compactenum}
    \item $N$ is a fixed-by-the-user virtual population size.
    \item In the diallelic condition, the first allele in the subscript is
    binned into $l$ (low = $1/N$), $m_2$ (mid=0.5), and $h_2$ (high= $(N-1)/N$) frequencies.
    \item In the 3-allelic condition, the first allele in the subscript is
    binned into $l$ (low = $1/N$), $m_3$ (mid=1/3), and $h_3$ (high= $(N-2)/N$) frequencies.
    \item In the 4-allelic condition, the first allele in the subscript is
    binned into $l$ (low = $1/N$), $m_4$ (mid=0.25), and $h_4$ (high= $(N-3)/N$) frequencies.
\end{compactenum}

This would yield 43 states:
\begin{compactenum}
    \item The 4 monomorphic states: $(N,0,0,0)$, $(0,N,0,0)$, $(0,0,N,0)$, and $(0,0,0,N)$.
    \item The 18 dimorphic states. For each of the 6 pairs of nucleotides,
    there would be 3 polymorphic bins such as:
    $(N-1, 1, 0, 0)$, $(N/2,N/2,0,0)$, and $(1, N-1, 0, 0)$.
    \item There are 16 states with 3 alleles. For each of the 4 triples of nucleotides,
    there would be 4 states:
    $(N-2, 1, 1, 0)$, $(1, N-2, 1, 0)$, $(1, 1, N-2, 0)$, and $(N/3, N/3, N/3, 0)$.
    \item There are 5 states with all 4 alleles:
    $(N-3, 1, 1, 1)$, $(1, N-3, 1, 1)$, $(1, 1, N-3, 1)$, $(1, 1, 1, N-3)$ and $(N/4, N/4, N/4, N/4)$.
\end{compactenum}

\subsection*{Parameterization}
This coarse binning breaks the easy connections with the population genetics underpinning.
However, we expect to have quite a bit of data, so we
could estimate an extra parameter to generate the state frequencies, and $Q$ matrix:
\begin{compactitem}
    \item The 3 free parameters needed to yield the 4 equilibrium frequency
    of the mutation process: $1 = \pi_A + \pi_C + \pi_G + \pi_T$
    \item The 5 free parameters needed to yield the 6 symmetric factors
    of the GTR rate matrix:  $\{r_{AC}$, $r_{AG}$, $r_{AT}$, $r_{CG}$, $r_{CT}$, $r_{GT}\}$
    \item Let $f_i$ be the equilibrium probability of their being in a state with $i$ alleles.
    So $1 = \sum_{i=1}^{4}f_i$.
    If $f_3 = f_4 = 0$ then $f_1, = 1- f_2$, this model reduces to the lmh-PoMo
    In the most general parameterization, we would introduce three free parameters for each of the
    non-monomorphic states.
    A promising simplification is to simply introduce 1 new such parameter:
    \begin{eqnarray*}
        f_2 & = & \phi \\
        f_3 & = & \phi^2 \\
        f_4 & = & \phi^3 \\
        f_1 & = & 1 - \phi - \phi^2 - \phi^3 \\
    \end{eqnarray*}
    where $0\leq \phi < 0.5436890126920764$ to allow for a non-zero probability of monomorphic condition.
    \item The relative frequency parameter, 
    $\psi = \pi_{IJ}^{(N/2)}/\pi_{IJ}^{(N-1)} = \pi_{IJ}^{(N/2)}/\pi_{IJ}^{(1)}$ is used as in lmh-PoMo.
    This govens the relative abundance of the mid-frequency bin states, given that 
    the state corresponds to a polymorphic condition.

    \item a symmetric rate factor: $\rho$ that governs the relative rate of drift between
    states that correspond to polymorphic for the same pair of nucleotides was introduced for the lmh-PoMo
    model.
    This probably does not need any extension of the general-lmh-PoMo
\end{compactitem}


\subsection*{Connectivity between polymorphic states.}
An ideal form of a general pomo model would have states corresponding to all possible ways of partitionining
    a population of virtual size $N$ into counts for each of the 4 nucleotides.
This would result in a huge number of states for the case of species that are polymorphic for all 4 nucleotides.
This seems very inefficient (given that we don't expect the species to spend much time in such states).

Even when moving to the coarse lmh binning, it would be intuitive to have a large number of states
    for the 3- and 4-allelic conditions.
For example, the $(N-1,1, 0,0) \rightarrow (N-2,1,1,0)$ should be modelled has having an instantaneous rate that
    is proportional to $(N-1)\pi_G r_{AG}$, while an $A\rightarrow G$ mutation in the middle-frequency state
    should occur at a rate of $\frac{N}{2}\pi_G r_{AG}$ and should correspond to a change like:
    $(N/2,N/2, 0,0) \rightarrow (N/2 - 1,N/2,1,0)$.
Unfortunately, this would lead to 10 tri-allelic states for every triple of nucleotides -- because there would need to be 
    a pair of states with very similar frequencies: $(N/2-1,N/2,1,0)$ and $(N/2,N/2 - 1,1,0)$.
For the 4-allelic condition, similar logic would require 23 states.

These large numbers of states for 3- and 4-allelic states seem unlikely to be helpful for phylogenetic inference.
It also seems implausible, that we would be able to reliably fit a lot of parameters that describe the transition 
    rates for these rare states.

So the proposal for the general lmh-pomo model has a much reduce state space.
The $(N-1,1, 0,0) \rightarrow (N-2,1,1,0)$ transition can still occur, but there are not states corresponding to 
``two-nucleotides are common and 1 is rare.''
This means that there is no obviously optimal way to treat mutations when the populartion is in a two-alleles at mid-frequency
    condition.

\subsubsection*{No mutations from the mid-range states}
One simple, though biologically unjustified, solution would be to prohibit the addition of third allele when in a state like $(N/2, N/2, 0, 0)$

This is the choice that will be fleshed out below.

\subsubsection*{Simultaneous mutation and drift}
Although violating the spririt of an ``instantaneous'' rate matrix, we
could allow mutations from $(N/2, N/2, 0, 0)$ to lead to $((N-2)/2, 1, 1, 0)$ and $(1, (N-2)/2, 1, 0)$ states.
This corresponds to a new mutation and fair amount of drift in one event.

One could also consider other mutation+drift events like $(N/2, N/2, 0, 0)\rightarrow(N/3, N/3, N/3, 0)$

\subsubsection*{State frequencies}

The equil. frequency of bein monorphic for state $I$is
\begin{equation}
    \pi_I f_1 = \pi_I \left(1 - \phi - \phi^2 - \phi^3\right)
\begin{equation}

The normalizing constant for comparing the frequencies of 2-allele states becomes:
\begin{equation}
    K_2 = \sum_I\sum_{J > I}\pi_I\pi_J r_{IJ}
\end{equation}
and the frequency of having $N-1$ copies of $I$ and 1 copy of $J$ 
is equal to $N-1$ copies of $J$ and 1 copy of $I$.
Both states are expected at:
\begin{equation}
\frac{\phi\pi_I\pi_J r_{IJ}}{K_2\left(2 + \psi\right)}.
\end{equation}
while the frequency of the state corresponding to $N/2$ copies of both $I$ and $J$ is:
\begin{equation}
\frac{\phi\pi_I\pi_J r_{IJ}\psi}{K_2\left(2 + \psi\right)}.
\end{equation}

Generalizing the idea that all 3 of the low-low-high combinations for a particular triplet of nucleotides ($I$, $J$, and $K$)
will have the same frequency we find:
\begin{equation}
    K_3 = \sum_I\sum_{J > I}\pi_I\pi_J\pi_K \left(r_{IJ}r_{IK} + r_{IJ}r_{JK} + r_{IK}r_{JK} \right)
\end{equation}
The low-low-high combination will occur at:
\begin{equation}
\frac{\phi^2\pi_I\pi_J\pi_K \left(r_{IJ}r_{IK} + r_{IJ}r_{JK} + r_{IK}r_{JK} \right)}{K_3\left(3 + \psi\right)}.
\end{equation}
and the mid-mid-mid combination will be at:
\begin{equation}
\frac{\phi^2\pi_I\pi_J\pi_K \left(r_{IJ}r_{IK} + r_{IJ}r_{JK} + r_{IK}r_{JK} \right)\psi}{K_3\left(3 + \psi\right)}.
\end{equation}

And for the 4-allele states, the low-low-low-high states occur at:
\begin{equation}
\frac{\phi^3}{\left(4 + \psi\right)}.
\end{equation}
and the all-four-equally-frequent state is at:
\begin{equation}
\frac{\phi^3 \psi}{\left(4 + \psi\right)}.
\end{equation}


\subsubsection*{Rates of interchange}
The transitions between the diallelic states and either the monomorphic or diallelic states are 
covered in the lmh-pomo doc.

For the diallelic, to 3-allelic, we demonstrate with one case (a new G allele for a population that had more A's than C):
\begin{eqnarray}
    Q_{N-1,1,0,0}^{N-2,1,1,0} & = & \frac{(N-1)\mu_{AG}}{N} \nonumber \\
     & = &  \frac{(N-1)\pi_G r_{AG}}{N} \\
     & = & \pi_{N-2, 1, 1, 0}r_{N-1,1,0,0}^{N-2,1,1,0} \nonumber\\
    r_{N-1,1,0,0}^{N-2,1,1,0} & = & \frac{(N-1)\pi_G r_{AG}}{N\pi_{N-2, 1, 1, 0}} \nonumber\\
            & = & \frac{(N-1)\pi_G r_{AG}K_3\left(3 + \psi\right)}{N\left[\phi^2\pi_A\pi_C\pi_G \left(r_{AC}r_{AG} + r_{AC}r_{CG} + r_{AG}r_{CG} \right)\right]} \nonumber\\
            & = & \frac{(N-1) r_{AG}K_3\left(3 + \psi\right)}{N\phi^2\pi_A\pi_C \left(r_{AC}r_{AG} + r_{AC}r_{CG} + r_{AG}r_{CG} \right)} \\
    Q_{N-2,1,1,0}^{N-1,1,0,0} & = & \pi_{N-1, 1, 0, 0}r_{N-1,1,0,0}^{N-2,1,1,0} \nonumber \\
    & = &  \left(\frac{\phi\pi_A\pi_C r_{AC}}{K_2\left(2 + \psi\right)}\right)\left(\frac{(N-1) r_{AG}K_3\left(3 + \psi\right)}{N\phi^2\pi_A\pi_C \left(r_{AC}r_{AG} + r_{AC}r_{CG} + r_{AG}r_{CG} \right)}} \right) \nonumber \\
     & = & \frac{(N-1)K_3\left(3 + \psi\right)r_{AG}r_{AC}}{NK_2 \left(2 + \psi\right)\phi\left(r_{AC}r_{AG} + r_{AC}r_{CG} + r_{AG}r_{CG} \right)}
\end{eqnarray}

Once again drift without mutation occurs between ``adjacent'' states with symmetric rate factor $\rho$:
\begin{eqnarray}
   Q_{N-1,1,0,0}^{N/2,N/2,0,0} = Q_{1,N-1,0,0}^{N/2,N/2,0,0} & = & \frac{\polyProb\pi_A\pi_G r_{AG}\psi \rho}{K_2 (2 + \psi)} \\
   Q_{N/2,N/2,0,0}^{N-1,1,0,0} = Q_{N/2,N/2,0,0}^{1,N-1,0,0} & = & \frac{\polyProb\pi_A\pi_G r_{AG} \rho}{K_2 (2 + \psi)}
\end{eqnarray}
for the 3-allele state:
\begin{eqnarray}
   Q_{\ast}^{N/3,N/3,N/3,0} & = & \frac{\phi^2\pi_A\pi_C\pi_G \left(r_{AC}r_{AG} + r_{CG}r_{AC} + r_{AG}r_{CG} \right)\psi\rho}{K_3\left(3 + \psi\right)}
\end{eqnarray}
\begin{eqnarray}
   Q_{\ast}^{N-2,1,1,0} & = & Q_{\ast}^{1,N-2,1,0} = Q_{\ast}^{1, 1, N-2,0} \nonumber \\
   & = & \frac{\phi^2\pi_A\pi_C\pi_G \left(r_{AC}r_{AG} + r_{CG}r_{AC} + r_{AG}r_{CG} \right)\rho}{K_3\left(3 + \psi\right)}
\end{eqnarray}
where $\ast$ represents any of the start legal start states (having all 3 of the alleles, but in different frequencies than the 
destination state)


For the 4-allelic mutation (illustrated with a 3-allelic (mainly A, some C, some G) state to the addition of T):
\begin{eqnarray}
    Q_{N-2,1,1,0}^{N-3,1,1,1} & = & \frac{(N-2)\mu_{AT}}{N} \nonumber \\
     & = &  \frac{(N-1)\pi_T r_{AT}}{N} \\
     & = & \pi_{N-3, 1, 1, 1}r_{N-2,1,1,0}^{N-3,1,1,1} \nonumber\\
    r_{N-2,1,1,0}^{N-3,1,1,1} & = & \frac{(N-2)\pi_T r_{AT}}{N\pi_{N-3, 1, 1, 1}} \nonumber\\
            & = & \frac{(N-2)\pi_T r_{AT}(4 + \psi)}{N\phi^3}\\
    Q_{N-3,1,1,1}^{N-2,1,1,0} & = & \pi_{N-2, 1, 1, 0}r_{N-2,1,1,0}^{N-3,1,1,1} \nonumber \\
    & = &  \left(\frac{\phi^2\pi_A\pi_C\pi_G \left(r_{AC}r_{AG} + r_{AC}r_{CG} + r_{AG}r_{CG} \right)}{K_3\left(3 + \psi\right)}\right)\left(\frac{(N-2)\pi_T r_{AT}(4 + \psi)}{N\phi^3}\right) \nonumber \\
    & = & \frac{(N-2)(4 + \psi)\pi_A\pi_C\pi_G \pi_Tr_{AT}\left(r_{AC}r_{AG} + r_{AC}r_{CG} + r_{AG}r_{CG} \right)}{N\phi K_3\left(3 + \psi\right)} 
\end{eqnarray}

The drift transitions for the 4-allele condition are:
\begin{eqnarray}
   Q_{\ast}^{N/4,N/4,N/4,N/4} & = & \frac{\phi^3\psi\rho}{4 + \psi}
\end{eqnarray}
\begin{eqnarray}
   Q_{\ast}^{N-3,1,1,1}  & = & \frac{\phi^3\rho}{4 + \psi}
\end{eqnarray}
where $\ast$ represents any of the start legal start states (having all 4 of the alleles, but in different frequencies than the 
destination state)


\bibliographystyle{splncs03}
\bibliography{pomo}




\end{document}

\begin{algorithm} \caption{}\label{}
\begin{algorithmic}
\end{algorithmic}
\end{algorithm}
