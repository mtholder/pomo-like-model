% This is LLNCS.DEM the demonstration file of
% the LaTeX macro package from Springer-Verlag
% for Lecture Notes in Computer Science,
% version 2.4 for LaTeX2e as of 16. April 2010
%
\documentclass{llncs}
%
\usepackage{amssymb}
\usepackage{amsmath}
\usepackage{pdfpages}
\usepackage{graphicx}
\usepackage{paralist}
\usepackage{xspace}
\usepackage{paralist}
\usepackage{bm}
\usepackage{algorithm,algorithmic}
\newcommand{\numOTUs}{\ensuremath{L}}
\newcommand{\numSites}{\ensuremath{K}}
\newcommand{\numSp}{\ensuremath{M}}
\newcommand{\numObsStates}{\ensuremath{k}}
\newcommand{\virtPopSize}{\ensuremath{N}}
\newcommand{\polyProb}{\ensuremath{\phi}}}
\newcommand{\Knorm}{\ensuremath{K}}}
\newcommand{\pomo}{PoMo\xspace}
% from http://tex.stackexchange.com/a/33547
\newcommand{\appropto}{\mathrel{\vcenter{
              \offinterlineskip\halign{\hfil$##$\cr
                      \propto\cr\noalign{\kern2pt}\sim\cr\noalign{\kern-2pt}}}}}
\DeclareMathOperator*{\argmax}{\arg\!\max}
\usepackage{hyperref}
\hypersetup{backref,  linkcolor=blue, citecolor=red, colorlinks=true, hyperindex=true}
%
\begin{document}
\title{Jukes-Cantor \pomo}
\titlerunning{JC-POMO implementation}
\author{Mark T.~Holder\inst{1,2}\and Alexandros Stamatakis\inst{1,3}, \and $\ldots$}
\authorrunning{Author 1 et al.} % abbreviated author list
%\tocauthor{Author 1, Author 2}
\institute{1. HITS gGmbH,\\2. Univ.~Kansas\\3. KIT}\\
%\email{\{Authors\}@h-its.org} \and Institute 2,\\ Address, Country\\
%\email{Author3@h-its.org}}
%% Eumerating parts in aligned environments %%
\newcommand\enum{\addtocounter{equation}{1}\tag{\theequation}}
\maketitle              % typeset the title of the contribution
\section {Motivating question}
Can one prove that a \pomo-style model (described below) leads (or does not lead)
to a statistically consistent estimator
of the phylogenetic tree from concatenated data in the ``anamoly zone''?

If so, does a \pomo-style model reduce part of parameter zone that leads to inconsistent
    tree estimation, or does is it consistent over the whole zone?

Can we say anything about the consistency of branch length estimation?

\subsection*{Approach}
(not much here, yet)

It seems like the most promising approach might be to work on
\begin{compactitem}
    \item a ``toy'' Jukes-Cantor version of the \pomo model (a dramatic simplification of the 
    \pomo model of \cite{DeMaioSK2013}),
    \item data consisting of long sequences one sampled haplotype from each species
    \item assumption of equal effective population sizes for all species.
\end{compactitem}

Consider a 10 state version with states: \texttt{A, C, G, T, AC, AG, AT, CG, CT, GT}.
With the first four representing fixed states of a species, and the
latter 6 representing the species being polymorphic for 2 alleles at a site.

State graph: Picture the 4 fixed states as points on a tetrahedron.
Rather than each edge of the tetrahedron being an edge in a graph, each corresponds
to a path of 2 edges that passes through the polymorphic state.

Rate exchangeabilities: 0 for non-adjacent states, and 1 for adjacent states.

Free parameters: a branch length, and a frequency of the polymorphism, \polyProb.

State frequencies: $(1-\polyProb)/4$ for the fixed states and $\polyProb/6$ for the polymorphic states.

$Q$ matrix for the state ordering given above is:
\begin{equation}
Q & = & \left(
\begin{matrix}
  - & 0 & 0 & 0 & \frac{\polyProb}{6} & \frac{\polyProb}{6} & \frac{\polyProb}{6} & 0 & 0 & 0\\
  0 & - & 0 & 0 & \frac{\polyProb}{6} & 0 & 0 & \frac{\polyProb}{6} & \frac{\polyProb}{6} & 0 \\
  0 & 0 & - & 0 & 0 & \frac{\polyProb}{6} & 0 & \frac{\polyProb}{6} & 0 & \frac{\polyProb}{6} \\
  0 & 0 & 0 & - & 0 & 0 & \frac{\polyProb}{6} & 0 & \frac{\polyProb}{6} &  \frac{\polyProb}{6} \\
  \frac{1-\polyProb}{4} & \frac{1-\polyProb}{4} & 0 & 0 & - & 0 & 0 & 0 & 0 & 0 \\
  \frac{1-\polyProb}{4} & 0 & \frac{1-\polyProb}{4} & 0 & 0 & - & 0 & 0 & 0 & 0 \\
  \frac{1-\polyProb}{4} & 0 & 0 & \frac{1-\polyProb}{4} & 0 & 0 & - & 0 & 0 & 0 \\
  0 & \frac{1-\polyProb}{4} & \frac{1-\polyProb}{4} & 0 & 0 & 0 & 0 & - & 0 & 0 \\
  0 & \frac{1-\polyProb}{4} & 0 & \frac{1-\polyProb}{4} & 0 & 0 & 0 & 0 & - & 0 \\
  0 & 0 & \frac{1-\polyProb}{4} & \frac{1-\polyProb}{4} & 0 & 0 & 0 & 0 & 0 & - 
 \end{matrix}
 \right)
\end{equation}

Mathematica can give an analytical result for the 9 distinct transition probabilities
but they are very involved.

If we use a single index for the fixed states and paired indices for poly states, we can express the 4 types of transitions starting in a fixed state as:
\begin{eqnarray}\Pr(i\rightarrow i) & = & \hskip 4em\mbox{stay in same fixed state}\\
\Pr(i\rightarrow j) & = & \hskip 4em\mbox{to diff fixed state}\\
\Pr(i\rightarrow ij) = \Pr(i\rightarrow ji) & = & \hskip 4em\mbox{to adjacent poly. state}\\
\Pr(i\rightarrow jk) & = & \hskip 4em\mbox{to non-adjacent poly. state}
\end{eqnarray}
(where $i$, $j$, $k$, and $l$ are all distinct indices).
Those equations are ordered by the minimum number of state-changing transitions that would have to occur
to cause the event.

Similarly the 5 types of transitions starting in a polymorphic state are
\begin{eqnarray}
\Pr(ij\rightarrow ij) & = & \hskip 4em\mbox{stay in same poly. state}\\
\Pr(ij\rightarrow i) & = & \hskip 4em\mbox{to adjacent fixed state}\\
\Pr(ij\rightarrow ik) & = & \hskip 4em\mbox{to neighbor poly. state}\\
\Pr(ij\rightarrow k) & = & \hskip 4em\mbox{to non-adjacent fixed state}\\
\Pr(ij\rightarrow kl) & = & \hskip 4em\mbox{to ``opposite'' poly. state}.
\end{eqnarray}
(where $i$, $j$, $k$, and $l$ are all distinct indices).

Note that $\polyProb$ controls the flux through the model (the rate of substitution) and
the relative frequency of the polymorphic states.

Probabilities at the tips of the tree.
If you only have 1 sample per species then the sample space of the 
    observation, $X$, for a species at a site is only the four nucleotides: \texttt{A, C, G, T}
The probability of observing the data, conditional on the state of the species, $s$, is:
\begin{eqnarray}
\Pr(X=i \mid s=i) & = & 1 \\
\Pr(X=i \mid s=j) & = & 0 \\
\Pr(X=i \mid s=ij)  = \Pr(X=j \mid s=ij) & = & 0.5 \\
\Pr(X=i \mid s=jk) & = & 0
\end{eqnarray}
(where $i$, $j$, and $k$ are all distinct indices).

\bibliographystyle{splncs03}
\bibliography{pomo}




\end{document}

\begin{algorithm} \caption{}\label{}
\begin{algorithmic}
\end{algorithmic}
\end{algorithm}
